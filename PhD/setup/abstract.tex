% !Mode:: "TeX:UTF-8" 

%====================================== 中文摘要 ==========================================
\BiAppendixChapter{摘~~~~要}{ABSTRACT (Chinese)}
\setcounter{page}{1}\pagenumbering{Roman}
\defaultfont

博士学位论文摘要正文为 1000 字左右。

内容一般包括:从事这项研究工作的目的和意义;完成的工作 (作者独立进行的研究工作及相应结果的概括性叙述);获得的主要结论 (这是摘要的中心内容)。博士学位论文摘要应突出论文的创新点。

摘要中一般不用图、表、化学结构式、非公知公用的符号和术语。

如果论文的主体工作得到了有关基金资助,应在摘要第一页的页脚处标注:本研究得到某某基金 (编号:) 资助。\footnote{本研究得到某某基金 (编号:) 资助。}


\vspace{\baselineskip}
\noindent{\fontsize{11.5pt}{11.5pt}\selectfont\bfseries 关\hspace{0.5em}键\hspace{0.5em}词}:西安交通大学;博士学位论文;\LaTeX{} 模板

{\color{red} 关键词由3~5个组成。关键词应从《汉语主题词表》中摘选,当《汉语主题词表》的词不足以反映主题时,可由申请人设计关键词,但须加注。每一关键词之间用分号分开,最后一个关键词后不打标点符号。由申请人设计的关键词,须在该关键词的右上角标注*,并在该页的页脚处注明“*表示非汉语主题词”。}

\vspace{\baselineskip}
\noindent{\fontsize{11.5pt}{11.5pt}\selectfont\bfseries 论文类型}:应用基础

{\color{red} 论文类型包括:a.理论研究(Theoretical Research);b.应用基础(Application Fundamentals);c.应用研究(Application Research);d.研究报告(Research Report);e.设计报告(Design Report);f.案例分析(Case Study);g.调研报告(Investigation Report);h.产品研发(Product Development);i.工程设计(Engineering Design);j.工程/项目管理(Engineering/Project Management);k.其它(Others)。}

\clearpage

%====================================== 英文摘要 ==========================================
\BiAppendixChapter{ABSTRACT}{ABSTRACT (English)}

\noindent 英文摘要正文每段开头不缩进,每段之间空一行。\newline

\noindent The abstract goes here. \newline

\noindent \LaTeX{} is a typesetting system that is very suitable for producing scientific and mathematical documents of high typographical quality.

%\noindent You will never want to use Word when you have learned how to use \LaTeX.

\vspace{\baselineskip}
\noindent{\fontsize{11.5pt}{11.5pt}\selectfont\bfseries KEY WORDS}: Xi'an Jiaotong University; Doctoral dissertation; \LaTeX{} template

\vspace{\baselineskip}
\noindent{\fontsize{11.5pt}{11.5pt}\selectfont\bfseries TYPE OF DISSERTATION}: Application Fundamentals

\clearpage{\pagestyle{empty}\cleardoublepage}% 目录从奇数页开始
